\documentclass[a4paper]{article}
\usepackage{fourier}
\usepackage[utf8]{inputenc}
\usepackage[T1]{fontenc}
\usepackage{a4wide}
\usepackage{amsmath,amsfonts,amssymb}
\usepackage{mathtools}

\newcommand{\CC}{\mathbb{C}}
\newcommand{\QQ}{\mathbb{Q}}
\newcommand{\RR}{\mathbb{R}}
\newcommand{\ZZ}{\mathbb{Z}}

\newcommand{\UHP}{\mathcal{H}}
\newcommand{\OO}{\mathcal{O}}
\newcommand{\End}{\mathrm{End}}

\DeclareMathOperator*{\trd}{trd}
\DeclareMathOperator*{\nrd}{nrd}

\newcommand{\cvtwo}[2]{{\left(\begin{array}{c}{#1}\\ {#2}\end{array}\right)}}
\newcommand{\mtwo}[4]{{\left(\begin{array}{cc}{#1} & {#2} \\ {#3} & {#4}\end{array}\right)}}

\title{Première vache}
\author{M. Rambaud\\ (notes de B. Smith)}
\date{28/11/2014}

\begin{document}
\maketitle

Ce qu'on veut faire:
\begin{enumerate}
    \item
        Parametrer une famille des surfaces abéliennes avec
        multiplication par une algèbre de quaternions...
    \item
        ...par une ``courbe de Shimura''...
    \item
        ...dont on veut calculer l'équation \(\pmod{p}\).
\end{enumerate}

\section{%%%%%%%%%%%%%%%%%%%%%%%%%%%%%%%%%%%%%%%%%%%%%%%%%%%%%%%%%%%%%%%%%%%%%%%
    Partie I
}%%%%%%%%%%%%%%%%%%%%%%%%%%%%%%%%%%%%%%%%%%%%%%%%%%%%%%%%%%%%%%%%%%%%%%%%%%%%%%%

\subsection*{(a) nos quaternions}
Soit \(B\)  une algèbre de quaternions sur \(\QQ\): c'est à dire,
\(B\) est une algèbre de dimension \(4\) sur \(\QQ\), telle que il existe
\(a\) et \(b\in\QQ\), tels que 
\[
    B = \QQ\langle i,j\rangle
    \ ,
    \quad \text{avec}\quad
    i^2 = a,
    \ 
    j^2 = b,
    \ 
    ij = -ji
    \ .
\]
On impose les conditions suivantes:
\begin{itemize}
    \item \(B\) est un corps non commutatif.
    \item \(B\) est ``indéfinie'': ie \(B\otimes_\QQ\RR \cong M_2(\RR)\).
    \item \(B\) est de discriminant \(D\), 
        où \(D = \prod_{p}p\) avec les premiers \(p\)
        tels que \(B\otimes_\QQ \QQ_p\) est un corps non commutatif
        (et pas \(\cong M_2(\QQ_p)\), etc).
\end{itemize}
Soit \(\OO\) un ``ordre maximal''
dans \(B\): c'est à dire, un réseau de rang \(4\) dans \(B\),
constituté d'éléments entiers sur \(\ZZ\), et tel que \(\OO\)
est un anneau.
On met
\[
    \OO^{(1)} 
    := 
    \{ \epsilon \in \OO : \nrd(\epsilon) = 1 \}
    \ .
\]

\subsection*{(b) construction des variétés abeliennes}
On fixe, une fois pour toutes, un isomorphisme
\[
    \psi: B \hookrightarrow B\otimes_\QQ\RR \cong M_2(\RR)
    \ .
\]
On a une action de \(M_2(\RR)\) sur \(\CC^2\):
\[
    \mtwo{a}{b}{c}{d} \cvtwo{w_1}{w_2}
    =
    \cvtwo{aw_1 + bw_2}{cw_2 + dw_2}
\]


Soit \(u = \cvtwo{w_1}{w_2} \in \CC^2\)
tel que \(w_1/w_2 \notin \RR\).
Propriété: l'orbite de \(u\) sous l'image de \(\OO\), qu'on note
\[
    \Lambda_u := \psi(\OO)u
    \ ,
\]
est un réseau de \(\CC^2\).
Alors, on a un tore complexe associé à \(u\):
\[
    A_u := \CC^2/\Lambda_u
    \ .
\]

\subsection*{(c) construction d'une ``forme de Riemann''}

Soit \(\mu \in B\) tel que \(\mu^2 = -1/D\)
(cf. exposés). On peut même choisir \(\mu^{-1}\in \OO\).
Soit 
\begin{align*}
    E : \Lambda_u\times\Lambda_u 
    & 
    \longrightarrow \ZZ
    \ ,
    \\
    (\psi(\alpha)u, \psi(\beta)u) 
    & 
    \longmapsto \trd(\mu\alpha\bar\beta)
\end{align*}
(où, pour \(\alpha = x + iy + jz + ijw\),
on a \(\bar\alpha := x - iy - jz - ijw\)
et \(\trd(\alpha) = 2x\)).

Soit \(E_\RR\) la forme \(\RR\)-bilinéaire alternée \(E\)
étendue à \(\CC^2\); alors
\begin{itemize}
    \item
        \(E(\sqrt{-1}\,\cdot,\sqrt{-1}\,\cdot) = E(\cdot,\cdot)\),
    \item
        \(E_\RR\) est entière sur le réseau \(\Lambda_u\), et
    \item
        ``la forme hermitienne associée à \(E_\RR\)
        est symétrique définite positive'';
\end{itemize}
\(\iff E(\sqrt{-1}\cdot,\cdot) > 0\),
ce qui implique que \(A_u\) est une variété algébrique.

\subsection*{(d) isomorphismes}

Soit \(\psi\) et \(\mu\) fixés.

À quelle condition
\((A_u,E_u,\psi)\) est isomorphe à \((A_v,E_v,\psi)\)?
Notons que après \(u = \cvtwo{w_1}{w_2} \mapsto \cvtwo{w_1/w_2}{1}\),
on peut supposer que 
\(u = \cvtwo{\tau}{1}\).
Alors:
\((A_\tau,E_\tau,\psi)\) isomorphe à \((A_{\tau'},E_{\tau'},\psi)\)
ssi 
\(\psi(\epsilon)\tau = \tau'\)
pour un inversible \(\epsilon\in\OO^{(1)}\),
où comme d'habitude
\(\mtwo{a}{b}{c}{d}\cdot\tau = \frac{a\tau + b}{c\tau + d}\).

\section{%%%%%%%%%%%%%%%%%%%%%%%%%%%%%%%%%%%%%%%%%%%%%%%%%%%%%%%%%%%%%%%%%%%%%%%
    Partie II
}%%%%%%%%%%%%%%%%%%%%%%%%%%%%%%%%%%%%%%%%%%%%%%%%%%%%%%%%%%%%%%%%%%%%%%%%%%%%%%%

\subsection*{(a) courbes de Shimura}

Le quotient \(X =  \psi(\OO^{(1)})\backslash\UHP\), une surface de
Riemann compacte,
paramètre les surfaces abéliennes \(A_\tau\) que l'on vient de
construire.

\subsection*{(b) l'anneau d'endomorphismes}

Théoreme: si \(A\) est une surface abélienne simple, 
alors \(\End(A)\otimes\QQ\) est l'un des suivants:
\begin{center}
    \begin{tabular}{rl|c}
                & \(\End(A)\otimes\QQ\) & dimension \\
        \hline
        (i)     & \(\QQ\) & 3 \\
        (ii)    & \(K\) corps quadratique imaginaire & 2 \\
        (iii)   & \(B\) quaternions & 1 \\
        (iv)    & ``corps quadratique CM'' & 0 \\
        (v)     & \(M_2(K)\), \(K\) quadratique imaginaire & 0 \\
    \end{tabular}
\end{center}
(Ici, ``dimension'' parle du sous-espace de l'espace de modules
correspondant.)
Nous sommes dans le cas (iii).

\section{%%%%%%%%%%%%%%%%%%%%%%%%%%%%%%%%%%%%%%%%%%%%%%%%%%%%%%%%%%%%%%%%%%%%%%%
    Partie III
}%%%%%%%%%%%%%%%%%%%%%%%%%%%%%%%%%%%%%%%%%%%%%%%%%%%%%%%%%%%%%%%%%%%%%%%%%%%%%%%



\end{document}
